\documentclass{article}
\usepackage{amsmath}
\usepackage[T1]{polski}
\usepackage{graphicx}
\title{Zadanie 3.2}
\author{Albert Kołodziejski}
\date{}
\begin{document}
\maketitle
\section*{Problem:}
Niech G będzie gramatyką
\[
    S \rightarrow aS|aSbS|\epsilon
\]
Udowodnić że
\[
    L(G) = \{ x: \textit{każdy przedrostek x ma co najmniej tyle symboli a, co symboli b}\}
\]
\section*{Rozwiązanie:}
Niech

\[
    A = \{ x: \textit{każdy przedrostek x ma co najmniej tyle symboli a, co symboli b}\}
\]
\subsection*{Pokażmy że $L(G) \subseteq A$}

Produkcja $aS$ dodaje nam z przodu jeden symbol $a$, Wystąpienie tej produkji nie zagraża wypadnięciem z języka $A$ bo dodanie kolejnego symbolu $a$ w dowolnym miejscu dowolnego przedrosta nie sprawi że ilośc symoli $a$ spadnie. \\
\\
Produkcja $aSbS$ dodaje nam jeden symbol $a$ i jeden symbol $b$ w tej kolejności, co oznacza że dla każdego symbolu $b$ występującego w dowolnym przedrostku musi wystąpić symbol $a$ przed nim w tym przedrostku.\\
\\
Ostatnią produkcją jest epsilon który oznacza że nie dodajemy żadnej litery, co również nie jest zagrożeniem dla wypadnięcia z języka $A$.\\
\\
Zaczynamy ze słowem które ma zero symboli $a$ i zero sybmoli $b$, należymy więc do języka $A$, używając dowolnej z możliwych produkji nie jesteśmy w stanie wypaść z języka $A$ więc $L(G) \subseteq A$.

\newpage

\subsection*{Pokażmy że $A \subseteq L(G)$}
Dla słów długości 0 jest to prawda, bo słowo puste można stworzyć za pomocą produkcji epsilon, należy więc do $L(G)$.\\
\\
Załóżmy że dla słów długości $n$ jest to prawda. Weźmy więc słowo $x$ takie że $|x| = n + 1$ oraz $x \in A$.\\
\\
Jeżeli słowo to składa się wyłącznie z samych symboli $a$, to możemy produkcji $aS$, n + 1 razy, a na koniec produkcji epsilon, co da nam słowo $x$. Słowo to więc należy do $L(G)$.\\
\\ 
Jeżeli słowo to ma co namniej jeden symbol $b$, to weźmy pierwszy występujący w tym słowie symbol $b$ i oznaczmy go przez $b_s$.\\
\\
Wiemy że nasze słowo $x$ musi zaczynać się symbolem $a$, bo w przeciwnym wypadku przedrostek o długości 1 posiadałby jeden symbol $b$ i zero symboli $a$, co oznaczałoby że słowo $x$ nie należy do $A$.\\
\\
Przed $b_s$ musi wystąpić symbol $a$, więc nasze słowo $x$ możemy przedstawić w postaci
\[
    x = yab_sz
\]
gdzie $y$ jest słowem składającym się wyłącznie z symboli $a$, $b_s$ jest pierwszym występującym symbolem $b$ w słowie $x$ a $z$ jest resztą symboli.\\
\\
Usunięcie $ab_s$ ze słowa $x$ nie sprawi że nasze słowo przestanie należeć do $A$, bo z każdego przedrostka $x$ zawierającego $ab_s$ usuwamy jednocześnie jeden symbol $a$ i jeden symbol $b$, co nie sprawi że wypadamy z języka $A$. Pozostałe przedrostki zawierają jedynie symbole $a$, więc usunięcie z nich jednego symbolu $a$ nie sprawi że wypadniemy z języka $A$.\\
\\
To oznacza że słowo $yz$ należy do $A$, co więcej $|yz| = n - 1$, więc na mocy założenia indukcyjnego słowo $yz$ należy do $L(G)$, a co za tym idzie jest generowane przez gramatyką $G$. Popatrzmy na sposób w jaki generujemy ostatni symbol $a$ w słowie $y$ podczas generowania słowa $yz$.\\
\\
Jeżeli użyliśmy do tego produkcji $aS$, to używając produkcji $aSbS$ gdzie pierwsze $S$ zastąpimy przez epsilon, otrzymamy $aab_sS$, generując słowo dalej tak jak byśmy generowali kolejne symbole $z$, otrzymamy słowo $x$.\\
\\
Jeżeli użyliśmy do tego produkcji $aSbS$, to używając produkcji $aSbS$ gdzie pierwsze $S$ zastąpimy przez epsilon, otrzymamy $aab_sSbS$, generując słowo dalej jak byśmy generowali kolejne symbole $z$, otrzymamy słowo $x$.\\
\\
Istnieje więc sposób na wygenerowanie $x$, więc $x \in L(G)$, $A \subseteq L(G)$. \\
\textbf{\boldmath{$A \subseteq L(G)$ oraz $L(G) \subseteq A$, więc $A = L(G)$.}}
\end{document}