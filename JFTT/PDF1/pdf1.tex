\documentclass{article}
\usepackage{amsmath}
\usepackage[T1]{polski}
\usepackage{graphicx}
\title{Zadanie 2.4}
\author{Albert Kołodziejski}
\date{}
\begin{document}
\maketitle
\section*{Problem:}
Czy język \textit{\(\{ 0^{n!} : n \in N\}\)} jest regularny?
\section*{Lemat o pompowaniu:}

Niech \textit{L} będzie językiem regularnym. Wtedy istnieje stała \textit{n} taka że jeśli \textit{z} jest dowolnym słowem z \textit{L} oraz \textit{\(|z| \ge n\)}, to \textit{z} możemy przedstawić w postaci \\
\textit{\(z = uvw\)}, gdzie \textit{\(|uv| \le n\)} i \textit{\(|v| \ge 1\)} oraz \textit{\(uv^iw\)} należy do \textit{L} dla każdego \textit{\(i \ge 0\)}.

\section*{Rozwiązanie:}
Zakładamy, że język ten jest regularny, oznacza to, że lemat o pompowaniu powinien zachodzić. Weźmy więc słowo \textit{\(z = 0^{n!}\)}, gdzie \textit{n} jest z lematu o pompowaniu.
\\
\\
Zgodnie z lematem o pompowaniu istnieją \textit{\(k \ge 0\)}, \textit{\(j \ge 1\)}, \textit{\(k + j \le n\)}, dla których słowo postaci:

\[
    0^k(0^j)^i0^{n! - k - j}
\]
należy do języka dla każdego \textit{\(i \ge 0\)}. Natomiast już dla \textit{\(i = 0\)} mamy:
\[
    |0^k(0^j)^{0}0^{n! - k - j}| = k + n! - k - j = n! - j < n!
\]
bo \textit{\(j \ge 1\)}
\[
    k + n! - k - j = n! - j \ge n! - n = (n-1)!(n-1) > (n-1)!
\]
Więc długość naszego słowa musi być mniejsza niż \textit{\(n!\)} oraz większa niż \textit{\((n - 1)!\)} a to oznacza, że nasze słowo nie należy do języka.
\\
\\
Otrzymaliśmy sprzeczność, więc dany język nie jest regularny.
\end{document}