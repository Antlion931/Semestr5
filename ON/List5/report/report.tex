% TODO:
% - [ ] Explain Ax = b
%    - [ ] Add that you do work only until last meaningful element
%    - [ ] Add that you can only check for rows that are in your block, or block below if you are in the last row 
\documentclass{article}
\usepackage{amsmath}
\usepackage{graphicx}
\usepackage[table]{xcolor}
\usepackage[T1]{fontenc}
\usepackage{algpseudocode}
\graphicspath{{images/}}
\title{List 5 report}
\author{Albert Kołodziejski}
\begin{document}
\maketitle
\section*{With out Partial Selection}
\subsection*{Data Structure}
We can safly assume that only cells that are colored can hold meaningful numbers when performing the algorithms. 
\begin{center}
    \begin{tabular}{| c|c|c | c|c|c | c|c|c |}
        \hline
        \cellcolor{blue!10} $a_{1,1}$ & \cellcolor{blue!10} $a_{1,2}$ & \cellcolor{blue!10} $a_{1,3}$ & \cellcolor{blue!10} $c_{1,4}$ & 0 & 0 & 0 & 0 & 0 \\
        \hline
        \cellcolor{red!10} $a_{2,1}$ & \cellcolor{red!10} $a_{2,2}$ &\cellcolor{red!10}  $a_{2,3}$ & \cellcolor{red!10} 0 & \cellcolor{red!10} $c_{2,5}$ & 0 & 0 & 0 & 0 \\
        \hline
        \cellcolor{green!10} $a_{3,1}$ & \cellcolor{green!10} $a_{3,2}$ & \cellcolor{green!10} $a_{3,3}$ & \cellcolor{green!10} 0 & \cellcolor{green!10} 0 & \cellcolor{green!10} $c_{3,6}$ & 0 & 0 & 0 \\
        \hline
        0 & 0 & \cellcolor{blue!25} $b_{4,3}$ & \cellcolor{blue!25} $a_{4,4}$ & \cellcolor{blue!25} $a_{4,5}$ & \cellcolor{blue!25} $a_{4, 6$} & \cellcolor{blue!25} $c_{4,7}$ & 0 & 0 \\
        \hline
        0 & 0 & \cellcolor{red!25} $b_{5,3}$ & \cellcolor{red!25} $a_{5,4}$ & \cellcolor{red!25} $a_{5,5}$ & \cellcolor{red!25} $a_{5, 6$} & \cellcolor{red!25} 0 & \cellcolor{red!25} $c_{5,8}$ & 0 \\
        \hline
        0 & 0 & \cellcolor{green!25}$b_{6,3}$ & \cellcolor{green!25} $a_{6,4}$ & \cellcolor{green!25} $a_{6,5}$ & \cellcolor{green!25} $a_{6, 6$} & \cellcolor{green!25} 0 & \cellcolor{green!25} 0 & \cellcolor{green!25} $c_{6,9}$ \\
        \hline
        0 & 0 & 0 & 0 & 0 & \cellcolor{blue!50} $b_{7,6}$ & \cellcolor{blue!50}$a_{7,7}$ & \cellcolor{blue!50}$a_{7,8}$ & \cellcolor{blue!50} $a_{7,9}$ \\
        \hline
        0 & 0 & 0 & 0 & 0 & \cellcolor{red!50}$b_{8,6}$ & \cellcolor{red!50}$a_{8,7}$ & \cellcolor{red!50}$a_{8,8}$ & \cellcolor{red!50}$a_{8,9}$ \\
        \hline
        0 & 0 & 0 & 0 & 0 & \cellcolor{green!50}$b_{9,6}$ & \cellcolor{green!50}$a_{9,7}$ & \cellcolor{green!50}$a_{9,8}$ & \cellcolor{green!50}$a_{9,9}$ \\
        \hline

    \end{tabular}
\end{center}
It is also worth meansoing that majority of work done by the algorittms is done for one row, then other and so on. That's why I propose to store every color as different Vector, so that There will be less cache misses. The consideretion here is to correcly interpret coordinates, as they are not the same as in the matrix.
\subsection*{Ax = b}
When developing algorithm I have considereted following optimazations:
\begin{itemize}
    \item{There will be none operations for last row}
    \item{when multiplaing }
\end{itemize}

\begin{algorithmic}
    for x in 1:(n-1)
            last_x = last_meaningful_index_in_row(A, x)
            for k in 1:(l-(x% l))
                y = x + k
\end{algorithmic}
\end{document}